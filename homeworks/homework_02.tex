\documentclass[11pt,nocut]{article}

\usepackage{../../latex_style/packages}
\usepackage{../../latex_style/notations}
\usepackage{fancyhdr}

\pagestyle{fancy}
\renewcommand{\headrulewidth}{1pt}
\fancyhead[R]{DSGA1014 - Fall 2021}
\fancyhead[L]{HW2 - Matrices \& linear transformations - due September 19, 2021 }

\setcounter{section}{2}

\begin{document}

% \input{../preamble_homeworks.tex}

\vspace{1cm}

\begin{problem}[2 points]
	Which of the following are linear transformations? Justify.
	\begin{enumerate}[label=\normalfont(\textbf{\alph*})]
		\item 
			$\displaystyle
			T: \left| 
			\begin{array}{ccc}
				\R^2 & \to & \R \\
				(x,y) & \mapsto & x-y
			\end{array}
		\right.
		$
	\item 
            $\displaystyle
            T: \left| 
            \begin{array}{ccc}
                \R^2 & \to & \R^2 \\
                (x,y) & \mapsto & (3x + y, x - xy)
            \end{array}
        \right.
        $
	\item 
			$\displaystyle
			T: \left| 
			\begin{array}{ccc}
				\R^{n \times n} & \to & \R^n \\
				A & \mapsto & \mathrm{diag}(A)
			\end{array}
		\right.\quad$ where $\mathrm{diag}(A)$ is the diagonal of the matrix $A$, defined by 
		$$\mathrm{diag}(A) = (A_{1,1}, \dots A_{n,n}) \,.$$
    \item 
        $\displaystyle
        T: \left| 
        \begin{array}{ccc}
            \R^{n \times n} & \to & \R^{n \times n}\\
            A & \mapsto & A^{-1}
        \end{array}
    \right.\quad$ defined on the set of homothety matrices $S = \{ \lambda \Id_n \, | \, \lambda \in \R^* \}$ (see slide 19 of lec 02), recall that $\R^*$ is the set of real without $0$. 
	\end{enumerate}
\end{problem}

\vspace{5mm}
\begin{problem}
	Consider a matrix $A \in R^{m \times n}$ and denote by $(c_1, \dots c_n)$ its columns (vectors in $\R^m$). Prove the equality of the sets 
	$$
	\Im(A)  = \Span(c_1, \dots, c_n).
	$$
	Recall that $\Im(A) \stackrel{\text{def}}{=}
	\{ Ax \, | \, x \in \R^n \}$.
\end{problem}


\begin{problem}[3 points] (Wait for after next lab to complete this problem! or have a look at the last few slides  we did not cover during class that are now completed on the website to find out what Gaussian elimination does.)
Let $$A = \left(\begin{matrix} 1 & 1 & 1 \\
	2 & 4 & 4 \\
	3 & 7 & k
	\end{matrix}\right)$$.
		\begin{enumerate}[label=\normalfont(\textbf{\alph*})]
		  \item Using Gaussian elimination  solve $Ax = 0$ and give a basis of $\Ker A$ (show all your steps). You will have to differentiate cases according to values of $k$.
		  \item Why does the system $A x = (1,2,3)$ has at least one solution (for any value of k)? Do not solve the system, use previous results of this HW to justify your answer. Find all values of $k$ for which the system $A x = (1,2,3)$ has infinitely many solutions. 
		  \item Find all values of $k$ for which the system $A x = (10, 1, 2017)$ has exactly one solution. Give this solution as a function of $k$.
	\end{enumerate}
\end{problem}

\begin{problem}[2 points]
	Let $B$ and $P$ be the matrices of in $\R^{3 \times 3}$:
	$$
	P = 
	\begin{pmatrix}
		0 & 1 & 0\\
		1 & 0 & 0\\
		0 & 0 & 1\\
	\end{pmatrix},
	\quad
	B = 
	\begin{pmatrix}
		B_{1,1} & B_{1,2} & B_{1,3}\\
		B_{2,1} & B_{2,2} & B_{2,3}\\
		B_{3,1} & B_{3,2} & B_{3,3}\\
	\end{pmatrix}
	$$
	with arbitrary entries for $B$.
	\begin{enumerate}[label=\normalfont(\textbf{\alph*})]
		\item Compute the matrix product $BP$. Why is $P$ called a permutation matrix?
		\item Compute $PB$. What can you notice?
	\end{enumerate}	
\end{problem}
\vspace{5mm}


\begin{problem}[$\star$]
	Let $A 
	= \begin{pmatrix}
		2 & -1 \\
		2 & -1
	\end{pmatrix}
	\in \R^{2 \times 2}$ 
	and $\displaystyle
	T: \left| 
	\begin{array}{ccc}
		\R^{2 \times 2} & \to & \R^{2 \times 2}\\
		M & \mapsto & AM
	\end{array}
\right.\quad$.
\begin{enumerate}[label=\normalfont(\textbf{\alph*})]
	\item Show that $T$ is linear. 
	\item Give a basis of $\Ker(T)$ and a basis of $\Im(T)$.
\end{enumerate} 
\end{problem}


\end{document}
