\documentclass[11pt,nocut]{article}

\usepackage{../../latex_style/packages}
\usepackage{../../latex_style/notations}
\usepackage{fancyhdr}

\pagestyle{fancy}
\renewcommand{\headrulewidth}{1pt}
\fancyhead[R]{DSGA1014 - Fall 2021}
\fancyhead[L]{HW6 - Eigenvalues/vectors \& Markov chains - due October 17, 2021 }


\setcounter{section}{6}

\begin{document}
% \maketitle

\input{../preamble_homeworks.tex}

\begin{problem}[2 points]
    
    \begin{enumerate}[label=\normalfont(\textbf{\alph*})]
        \item Show that 6 is an eigenvalue for the matrices $A$ and $B$ in $\R^3$ defined below. In each case give one associate eigenvector.
        $$
        A = \begin{pmatrix}
            1 & 2 & 3\\
            2 & 3 & 1\\
            3 & 1 & 2
        \end{pmatrix} , \quad
        B = \begin{pmatrix}
            1 & 2 & 3\\
            4 & 0 & 2\\
            4 & 5 & -3
        \end{pmatrix} 
        $$
        \item Let $A \in \R^{n \times n}$ be a square matrix such that the sum of its rows is equal. In other words, there exists $\mu \in \R$ such that for any integer $i$ $(1 < i \leq n)$, $\sum_{j=1}^n A_{i,j} = \mu$. Show that $A$ admits one pair of eigenvector-eigenvalue and give their values. 
    \end{enumerate}
\end{problem}

\vspace{1cm}

\begin{problem}[2 points]
    For a square symmetric matrix, we call eigen decomposition the collection of all its eigenvector-eigenvalue pairs. Let $A\in \R^{n\times n}$ be a square symmetric matrix. Give the eigen decomposition of $A^k$ as a function of the eigen demcomposition of $A$ for any integer $k > 0$.
\end{problem}

\vspace{1cm}

\begin{problem}[2 points]
    The trace of a square $A\in \R^{n\times n}$ matrix is defined as the sum of its diagonal elements
    $$
    \mathrm{Tr}(A) = \sum_{i=1}^n A_{i,i}.
    $$
    \begin{enumerate}[label=\normalfont(\textbf{\alph*})]
        \item Show that for any $B \in R^{n \times m}$ and $C \in \R^{m \times n}$, $\mathrm{Tr}(BC) = \mathrm{Tr}(CB)$.
        \item Use the previous result to show that for a symmetric matrix $A\in \R^{n\times n}$, its trace is equal to the sum of its eigenvalues.
    \end{enumerate}
\end{problem}

\vspace{1cm}

\begin{problem}[3 points]
    Consider a Washington square squirell trapped in a box divided in 9 rooms. At any point of time, the squirell decides to go through any of the available doors or stay in the room, all actions with equal probability. Use \texttt{numpy} or any other programming language to help you solve this problem.\\
    
    \vspace{0.3cm}

    \begin{minipage}{0.2\textwidth}
        \includegraphics[width=\textwidth]{box.pdf}
    \end{minipage}
    \begin{minipage}{0.75\textwidth}
        \begin{enumerate}[label=\normalfont(\textbf{\alph*})]
            \item Construct the transition stochastic matrix $P$ for this problem. Can you find an integer $k \leq 1$ such that $P^k$ has only strictly positive entries?
            \item Find the invariant measure for this problem.
            \item Using a symmetry argument, show that you can also solve this problem using a $ 3 \times 3$ matrix.
        \end{enumerate}
    \end{minipage}
        
    % \end{figure}
\end{problem}

\vspace{1cm}

\begin{problem}[$\star$] A symmetric matrix $M \in \R^n$ is positive semi-definite if, for any $x \in \R^n$,
    $$x^\top M x \geq 0.$$
    We say furthermore that $M$ is postive definite if $x^\top M x > 0$ for any non-zero vector.
	\begin{enumerate}[label=\normalfont(\textbf{\alph*})] 
		\item Show that $M$ is positive semi-definite if and only if its eigenvalues are non-negative.
		\item Give a necessary and sufficient condition on the spectrum of $M$ for the matrix to be definite positive.
    \end{enumerate}
\end{problem}


% \centerline{\pgfornament[width=7cm]{87}}

%\bibliographystyle{plain}
%\bibliography{./references.bib}
\end{document}
